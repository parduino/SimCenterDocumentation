\documentclass{simcenterdocumentation}
\usepackage[backend=biber]{biblatex}
%\usepackage{subfig}
\usepackage{subcaption}
\usepackage{multirow}
\usepackage{adjustbox}
\usepackage{cleveref}
\usepackage{longtable}

\usepackage{hhline}
\usepackage{color, colortbl}
\definecolor{lightgray}{HTML}{E5E4E2}

%% SETTING ENVIRONMENT FOR PYTHON CODE SNIPPETS %%%%%%%%%%%%%%%%%%%%%%%%%%%%%%% 
\usepackage[utf8]{inputenc}

\graphicspath{{../Common/}{.}} %Setting the graphicspath
\makeatletter % Search additional directories for inputs
\def\input@path{{../Common/}{.}}
%or: \def\input@path{{/path/to/folder/}{/path/to/other/folder/}}
\makeatother

%%%%%%%%%%%%%%%%%%%%%%%%%%%%%%%%%%%%%%%%%%%%%%%%%%%%%%%%%%%%%%%%%%%%%%%%%%%%%%% 

% To compile this file, run "latex/pdflatex codedoc", then "biber codedoc"
% (or "bibtex codedoc", if the output from latex asks for that instead),
% and then "latex/pdflatex codedoc" (without the quotes in each case).

% Double spacing, if you want it.  Do not use for the final copy. Can also specify
% draft as a document class option. This will generate double spacing and placeholders
% for title page and header images
%% \def\dsp{\def\baselinestretch{2.0}\large\normalsize}
%% \dsp

\bibliography{../Common/references}


\begin{document}
% Declarations for Front Matter
% Software title followed by optional second line
\title{Requirements Traceability Matrix (RTM)}
% Use superscripts to indicate author affiliations
%\author{Frank McKenna}
\institutions{NHERI SimCenter, UC Berkeley}
\softwarename{RTM}
\softwareversion{1.2}
\softwarepage{https://simcenter.designsafe-ci.org/research-tools/pbe-application/}

%%% DON'T MESS WITH THESE SETTINGS %%%%%%%%%%%%%%%%%%%%%%%%%%%%%%%%
\hypersetup{pageanchor=false}
\maketitle
%\copyrightpage
%\acknowledgments

\hypersetup{pageanchor=true}
\begin{frontmatter}

\pagestyle{plain}
{
  \renewcommand{\thispagestyle}[1]{}
  \tableofcontents
  \clearpage
%  \listoffigures
% \clearpage
  \listoftables
}


\end{frontmatter}
\pagestyle{somewhatsimple}
%%%%%%%%%%%%%%%%%%%%%%%%%%%%%%%%%%%%%%%%%%%%%%%%%%%%%%%%%%%%%%%%%%%
% Create separate tex files for each chapter and provide them as inputs


\chapter{Introduction}
\label{chap:about}

The Requirements Traceability Matrix (RTM) is presented as tables linking requirements with project deliverables.  The requirements for the SimCenter have been obtained from a number of sources:
\begin{enumerate}
\item GC: Grand challenges in hazard engineering are the problems, barriers, and bottlenecks that hinder the ideal of a nation resilient from the effects of natural hazards. The vision documents referenced in the solicitation [2, 3, 5, 6] outline the grand challenges for wind and earthquake hazards. These documents all present a list of research and educational advances needed that can contribute knowledge and innovation to overcome the grand challenges. The advances summarized in the vision documents were identified through specially formed committees and workshops comprising researchers and practicing engineers. They identified both the grand challenges faced and also identified what was needed to address these challenges. The software needs identified in these reports that are applicable to research in natural hazards as permitted under the NSF NHERI program were identified in these reports. Those tasks that the NHERI SimCenter identified as pertaining to aiding NHERI researchers perform their research and those which would aid practicing engineers utilize this research in their work are identified here.
\item SP: From the senior personnel on the SimCenter project. The vision documents outline general needs without going into the specifics. From these general needs the senior personnel on the project  identified specific requirements  that would provided a foundation to allow research.
\item UF: SimCenter workshops, boot camps and direct user feedback. As the SimCenter develops and releases tools, feedback from researchers using these tools is obtained at the tool training workshops, programmer boot-camps,  in one-on-one discussions, via direct email, and through online user feedback surveys. 
\end{enumerate}  

\chapter{Software Requirements}
\label{chap:requirements}

The software requirements are many. For ease of presentation they are broken into two groups:
\begin{enumerate}
\item Regional Scale - Activities to allow researchers to examine the resilience of a community to natural hazard events.
\item Building Scale - Activities to allow researchers to improve on methods related to response assessment and performance based design of individual buildings subject to the impact of a natural hazard.
\end{enumerate}

\section{Regional Scale}


\newcolumntype{C}[1]{>{\centering\arraybackslash}p{#1}}

\begin{longtable}{|  C{.07\textwidth} | p{.5\textwidth} | C{.08\textwidth} | C{.09\textwidth} | C{.08\textwidth} | C{.08\textwidth} |}

\caption{Requirements for Regional Simulations aiding Community Resilience}
\label{tab:regionalRequirements} 
 \\
   \hline
\rowcolor{lightgray}

\textbf{\#} & \textbf{Description} & \textbf{SRC} & \textbf{WBS} & \textbf{PRI} & \textbf{VER} \\ \hline
R1 & Ability to perform regional simulation allowing communities to evaluate resilience and perform what-if types of analysis for natural hazard events & GC  & 1.3.10 rWhale & M & 1.0 \\  \hline
R1.1 & Perform such simulations for ground shaking due to Earthquake & GC & 1.3.10 rWhale & M & 1.0  \\ \hline
R1.2 & Ability to perform such simulations for wave action due to Earthquake induced Tsunami  & GC &  1.3.10 rWhale &  M &   \\  \hline
R1.3 & Ability to perform such simulations for wind action due to Hurricane & GC & 1.3.10 rWhale & M & \\  \hline
R1.4 & Ability to perform such simulations for wave action due to Hurricane Storm Surge & GC & R1.3.10 rWhale & M & \\ \hline
R1.5 & Ability to perform such for multi-hazard simulations: wind + storm surge, rain, wind and water borne debris & GC & 1.3.10 rWhale & M & \\ \hline
R1.6 & Ability to incorporate damage to lifelines in determination of community resilience & GC & 1.3.10 rWhale & M &  \\ \hhline{======}


R2.1 & Ability of stakeholders to perform simulations of different scenarios that aid in planning and response after damaging events & GC & 1.3.10 rWhale & M & 1.0 \\ \hline
R2.2 &  Ability to utilize HPC resources in regional simulations that enables repeated simulation for stochastic modeling & GC & 1.3.10 rWhale & M & 1.0 \\ \hline
R2.3 &  Provide open-source software for developers to test new data and algorithms & GC & 1.3.10 rWhale & M & 1.0  \\ \hline
R2.4 & Ability to use a tool created by linking heterogeneous array of simulation tools to provide a toolset for regional simulation & GC & 1.3.10 rWhale & M & 1.0 \\ \hline
R2.5 &  Ability to utilize existing open-source software for faster deployment & GC & 1.3.10 RDT & M & 1.0 \\ \hline
R2.6 &  Ability to utilize ensemble techniques  & GC & 1.3.10 rWhale & M & 1.0 \\ \hline
R2.7  & Ability to include multi-scale nonlinear models & GC & 1.3.10 rWhale & M & 1.0 \\ \hline
R2.8 & Ability to include a formal treatment of uncertainty and randomness & GC & 1.3.10 rWhale & M & 1.0 \\ \hline
R2.9 & Ability to include latest information and algorithms (i.e. new attenuation models, building fragility curves, demographics, lifeline performance models, network interdependencies, indirect economic loss)
& GC & rWhale 1.3.10 & D & \\ \hline
R2.10 &  Ability to use GIS so communities can visualize hazard impacts & GC & 1.3.10 rWhale & M & \\ \hline
R2.11 &  Ability to explore different strategies in community development, pre-event, early response, and post event, through long term recovery & GC & 1.3.10 RDT & P & \\ \hline
R2.12 &  Ability to use system that creates and monitors real-time data, updats models, incorporates crowdsourcing technologies, and informs decision makers & GC & 1.3.10 RDT & P & \\ \hline
R2.13 &  Ability to use sensor data to update models for simulation and incorporate sensor data into simulation & GC & 1.3.10 RDT & P & \\  \hhline{======}



R3.1& Ability to use open-source version of Hazus & GC & 1.3.6 pelicun & M & 1.0 \\ \hline
R3.2 &  Ability to incorporate improved damage and fragility models for buildings and lifelines & GC & 1.3.6 pelicun & M & 1.0 \\ \hline
R3.3 &  Ability to incorporate improved indirect economic loss estimation models & GC & 1.3.6 pelicun & M & \\ \hline
R3.4 & Ability to include demand surge in determination of damage and loss estimation & GC & 1.3.6 pelicun & M & \\ \hline
R3.5 & Ability to include lifeline disruptions & GC & 1.3.6 pelicun & M & \\ \hhline{======}


R4.1 & Promote 'living' community risk models utilizing local inventory data from various scources & GC & 1.3.0 rWhale & M & \\ \hline
R4.2 & Ability to use cumulative knowledge bases rather than the piecemeal individual approaches & GC & 1.3.3 & M & \\ \hline
R4.3 & Developing and sharing standardized definitions, measurement protocols and strategies for data collection & GC & 1.3.3 & M & \\ \hline
R4.3 & Developing tools that utilize GIS information and online images, e.g. google maps, for data collection & GC & 1.3.4 BRAILS & M & 1.0 \\ \hline
R4.3.1 & Predicting if building is a soft-story building for earthquake simulations & UF & 1.3.4 BRAILS & M & 1.0 \\ \hline
R4.3.2 & Predicting roof shape of building for hurricane wind simulation & SP & 1.3.4 BRAILS & M & 1.0 \\ \hline
R4.3.3 & Predicting level first floor of occupancy for hurricane storm surge simulation & SP & 1.3.4 BRAILS & M &  \\ \hline
R4.4 & Providing instruction on gathering information from WWW for purposes of these regional simulations & UF & 1.2.3 Summer Bootcamp & M & V2.0 \\ \hline
R4.5 & Developing, sharing, and archiving datasets for analyzing and modeling resilience and vulnerability over time & GC & 1.3.3 & M & \\  \hline
R4.6 & Ability to use GIS, high resolution elevation and soil data for wind and storm surge simulations & GC & 1.3.10 & D &  \\ \hline 
R4.7 & Ability to use validated multi-scale models (materials, components, elements) of built environment & GC & 1.3.3 & D & \\ \hline 
R4.8 & Ability to use a national database of models for hazard, buildings, and lifelines created for multiple hazards & GC & P & 1.3.3 & \\ \hhline{======}



5.1  &  Ability to perform validation studies to calibrate accuracy of models & GC & 1.3.10 rWhale & M &    \\ \hline
5 & Identify knowledge gaps and promote NSF generated knowledge through regional demonstration projects that help generate linkages to operational entities and decision makers & GC & 1.4.2 & M &  \\ \hline
5.1 & Provide Earthquake Testbeds& SP & 1.4.2 Testbeds & M & 1.0 \\ \hline
5.1.1 & Provide Bay Area Earthquake Testbed & SP & 1.4.2 Testbeds & M & 1.0 \\ \hline
5.1.2 & Provide Anchorage Earthquake Testbed & SP & 1.4.2 Testbeds & M & 1.1 \\ \hline5
.2 & Provide Tsunami Testbed & SP & 1.4.2 Testbeds & M & \\ \hline
5.3 & Provide Atlantic City Hurricane Wind Testbed & SP & 1.4.2 Testbeds & M &  \\ \hline
5.4 & Provide Atlantic City Hurricane Wind and Storm Surge Testbed & SP & 1.4.2 Testbeds & M & \\ \hline
5.5 & Provide Earthquake and Lifelines Testbed & SP & 1.4.2 Testbeds & M & \\ \hline

\end{longtable}

\noindent
KEY:\\
Source (SRC): GC=Needed for Grand Challenges, SP=Senior Personnel, UF=User Feedback \\
Work Breakdown Structure (WBS): SimCenter WBS  Number \\
Priority (PRI): M=Mandatory, D=Desirable, P=Possible Future \\
Version (VER): Version number the basic requirement was first met 


\clearpage
\section{Building Scale}

For building scale simulations, the requirements are broken down by SimCenter application. There are a number of applications under development for each of the hazards. Many of the requirements related to UQ and nonlinear analysis are repeated amongst the different applications under the assumption that if they are beneficial to engineers dealing with one hazard, they will be beneficial to engineers dealing with other hazards.

\subsection{Response of Building to Wind Hazard}
The following are the requirements for response of single structure due to wind action. The requirements are being met by the WE-UQ application. All requirements in this section are related to work in WBS 1.3.7.

 \begin{longtable}{| p{.09\textwidth} | p{.6\textwidth} | p{.08\textwidth} | p{.08\textwidth} |  p{.08\textwidth} |}

\caption{Requirements for WE-UQ}
  \label{tab:featureRequirements}  
    \\
   \hline
\rowcolor{lightgray}
\# & Description & Source & Priority & Version \\ \hline

W1 & \textbf{Ability to determine response of Building Subject to Wind Loading including formal treatment of randomness and uncertainty uncertainty} & GC & M & 1.0  \\ \hline
W1.1 & Ability of Practicing Engineers to use multiple coupled resources (applications, databases, viz tools) in engineering practice & GC & M & 1.0  \\ \hline
W1.2 & Ability to utilize resources beyond the desktop including HPC & GC & M & 1.0 \\ \hline
W1.3 & Tool available for download from web & GC & M & 1.0 \\ \hline
W1.4 & Ability to obtain training and education with respect to interaction of structure and wind to ensure research is appropriately applied  & GC & M & \\ \hhline{=====}
W2 & \textbf{Ability to select from different Wind Loading Options } & SP & M & 1.0 \\ \hline
W2.1 & Utilize Extensive wind tunnel datasets in industry and academia for wide range of building shapes & GC & M & 2.0 \\ \hline
W2.1.1 & Accommodate Range of Low Rise building shapes & SP & M &  \\ \hline
W2.1.1.1 & Flat Shaped Roof - TPU dataset & SP & M & 2.0 \\ \hline
W2.1.1.2 & Gable Shaped Roof - TPU dataset & SP & M & \\ \hline
W2.1.1.3 & Hipped Shaped Roof - TPU dataset & SP & M & \\ \hline
W2.1.2 & Accommodate Range of High Rise building  & SP & M & 1.0 \\ \hline
W2.1.2.1 & Interface with Vortex Winds DEDM-HRP Web service & SP & M & 1.0 \\ \hline
W2.1.3 & Accommodate Data from Wind Tunnel Experiment & SP & M & 2.0 \\ \hline
W2.1.3.1 & Cuboid - User Provided Wind Tunnel Experiment Data  & SP & M & 2.0 \\ \hline
W2.2 & Computational Fluid Dynamics tool for utilizing open source CFD software for practitioners & GC & M & 1.1 \\ \hline
W2.2.1 & Simple CFD model generation and turbulence modeling & GC & M & 2.0 \\ \hline
W2.2.2 & Uncoupled OpenFOAM CFD model with nonlinear FEM code for building response & SP & M & 1.1 \\ \hline
W2.2.3 & Coupled OpenFOAM CFD model with nonlinear FEM code for building response & SP & M &  \\ \hline
W2.3 & Quantification of Effects of Wind Borne Debris & GC & D & \\ \hline
W2.4 & Application to utilize GIS and online to account for wind speed given local terrain, topography and nearby buildings & GC & D & \\ \hline
W2.5 & Ability to utilize synthetic wind loading algorithms & SP & M & 1.0  \\ \hline
W2.5.1 & per Wittig and Sinha & SP & D & 1.0  \\ \hhline{=====}
W3 & \textbf{Ability to select different Building Model Generators} & GC & M & 2.0 \\ \hline
W3.1 & Ability to quickly create a simple nonlinear building model & GC & D & 1.0 \\ \hline
W3.2  & Ability to define building and use Expert System to generate FE mesh & SP & &  \\ \hline
	3.2.1 & Expert system for Concrete Shear Walls & SP & M &  \\ \hline
	3.2.2 & Expert system for Moment Frames & SP & M &  \\ \hline
	3.2.3 & Expert system for  Braced Frames & SP & M &   \\ \hline
W3.3 & Ability to define building and use Machine Learning applications to generate FE & GC &  &  \\ \hline
	W3.3.1 & Machine Learning for Concrete Shear Walls & SP & M &  \\ \hline
	W3.3.2 & Machine Learning for Moment Frames & SP & M &  \\ \hline
	W3.3.3 & Machine Learning for Braced Frames & SP & M &   \\ \hline
	W3.4 & Ability to specify connection details for member ends & SP & M & 2.2 \\ \hline
	W3.5 & Ability to define a user-defined moment-rotation response representing the connection details & SP & D & 2.2 \\ \hline
	W4 & \textbf{Ability to Perform Nonlinear Structural Analysis} & GC & M & 1.0 \\ \hhline{=====}
W4.1 & Ability to use utilize existing nonlinear analysis software used in earthquake engineering & GC & M & 1.0 \\ \hline
W4.1.1 & Utilize open source OpenSees software & SP & M & 1.0 \\ \hline
W4.2.1 & Ability to provide own OpenSees Analysis script to OpenSees engine. & SP & D & 1.0 \\ \hline
W4.3.1 & Ability to provide own Python script and use OpenSeesPy engine. & SP & O & 1.2 \\ \hline
W4.2 & Ability to use alternative FEM engine & SP & M & 2.0 \\ \hhline{=====}

UQ & \textbf{Ability to use various UQ Methods} & GC & M &  \\ \hline
UQ1 & \textbf{Forward Propagation Methods} & GC  & M & 1.0 \\ \hline
UQ1.1 & Ability to use basic Monte Carlo and LHS methods & SP & M & 1.0 \\ \hline
UQ1.2 & Ability to use Importance Sampling  & SP & M & 2.0 \\ \hline
UQ1.3 & Ability to use Gaussian Process Regression & SP & M & 2.0 \\ \hline
UQ1.4 & Ability to use Own External UQ Engine & SP & M &  \\ \hline
UQ2 & \textbf{Ability to use various Reliability Methods} & UF & M & 1.0 \\ \hline
UQ2.1 & Ability to use First Order Reliability method & UF & M &  \\ \hline
UQ2.2 & Ability to use Second Order Reliability method & UF & M & \\ \hline
UQ2.2 & Ability to use Surrogate Based Reliability & UF & M & \\ \hline
UQ2.3 & Ability to use Own External Application to generate Results & UF & M &  \\ \hline
UQ3 & \textbf{Ability to user various Sensitivity Methods} & UF & M & 1.0  \\ \hline
UQ3.1 & Ability to obtain Global Sensitivity Sobol's indices & UF & M &  \\ \hline
UQ4 & \textbf{Various Random Variable Probability Distributions} & SP & M & 1.0 \\ \hline
UQ4.1 & Ability to Define Variables of different types: & SP & M & 1.0  \\ \hline
UQ4.1.1 & Normal & SP & M  & 1.0 \\ \hline
UQ4.1.2 & Lognormal & SP & M & 1.0 \\ \hline
UQ4.1.3 & Uniform & SP & M & 1.0  \\ \hline
UQ4.1.4 & Beta & SP & M & 1.0 \\ \hline
UQ4.1.5 & Weibull &  SP & M  & 1.0 \\ \hline
UQ4.1.6 & Gumbel &  SP & M & 1.0  \\ \hline
UQ4.2 & User defined Distribution & SP & M &  \\ \hline
UQ4.3 & Define Correlation Matrix & SP & M &  \\ \hline
UQ4.4 & Random Fields & SP & M &  \\ \hline
>>>>>>> b55c16b9402fc3444addf52182320452de7e27c2




    W8 & \textbf{Ability to obtain Application Outputs} &  & & \\ \hline
    W8.1 & Ability to see pressure distribution on building & GC & M &   \\ \hline
    W8.2 & Ability to obtain basic building responses & SP & M &   \\ \hline
    W8.3 & Ability to Process own Output Parameters & UF & M & 1.1  \\ \hhline{=====}
    WE & \textbf{Ability to obtain Educational Materials} &  &  & \\ \hline
    WE1 & Ability to obtain training to ensure the research is appropriately applied & GC & M & \\ \hline 
    WE2 & Documentation exists on tool usage & SP & M & 1.0  \\ \hline
    WE3 & Video Exists demonstrating usage & SP & M &  \\ \hline
    WE4 & Verification Examples Exist & SP & M &   \\ \hline
    WE4 & Validation Examples Exist, validated against tests or other software & GC & M &  \\\hhline{=====}
    WM & \textbf{Misc.} &  &  & \\ \hline
    WM1 & Tool to allow user to load and save user inputs & SP & M & 1.0 \\ \hline
    WM2 & Installer which installs application and all needed software & UF & M &   \\ \hline
	\bottomrule 
               
\end{longtable}

\noindent
KEY:\\
Source: GC=Needed for Grand Challenges, SP=Senior Personnel, UF=User Feedback \\
Need: M=Mandatory, D=Desirable, P=Possible Future \\
Version: Version number the basic requirement was met 



 
 \clearpage
 \subsection{Response of Building to Hydrodynamic Effects Due to Tsunami or Coastal Inundation}
The following are the requirements for response of single structure due to hydrodynamic effects of water caused earthquake induced tsunami or coastal inundation due to a Hurricane.. The requirements are being met by the Hydro-UQ application. All requirements in this section are related to work in WBS 1.3.7.

 \begin{longtable}{| p{.07\textwidth} | p{.65\textwidth} | p{.08\textwidth} | p{.08\textwidth} |  p{.08\textwidth} |}
            
                               \caption{Requirements for HydroUQ} 
\label{tab:featureRequirements}    
 \\
   \hline
\rowcolor{lightgray}
       \# & Description & Source & Priority & Version \\ \hline
       
H1 & \textbf{Ability to determine response of Building Subject to Wave Loading due to Tsunami and Coastal inundation including formal treatment of randomness and uncertainty uncertainty} & GC & M &  \\ \hline
H1.1 & Simulation of overland flow including waves, debris, flood velocity, erosion at building, and channeling effects & GC & M & \\ \hline
H1.2 & Use CFD to model interface and impact between water loads and buildings & GC & M & \\ \hline
H1.3 & Ability to quantify effect of flood borne debris & GC & M & \\ \hline
H1.4 & Ability of Practicing Engineers to use multiple coupled resources (applications, databases, viz tools) in engineering practice & GC & M & \\ \hline
H1.5 & Ability to utilize resources beyond the desktop including HPC & GC & M & \\ \hline
H1.6 & Ability to obtain training and education with respect to interaction of structure and water to ensure research is appropriately applied  & GC & M & \\ \hline

\bottomrule 
             
\end{longtable}

\noindent
KEY:\\
Source: GC=Needed for Grand Challenges, SP=Senior Personnel, UF=User Feedback \\
Need: M=Mandatory, D=Desirable, P=Possible Future \\
Version: Version number the basic requirement was met 



 
 \clearpage
\subsection{Response of Building to Earthquake Hazard}
The following are the requirements for response of single structure to earthquake hazards. The requirements are being met by the EE-UQ application. AAll requirements in this section are related to work in WBS 1.3.8.

 \begin{longtable}{| p{.07\textwidth} | p{.65\textwidth} | p{.08\textwidth} | p{.08\textwidth} |  p{.08\textwidth} |}

  \caption{Requirements for EE-UQ}
  \label{tab:featureRequirements}    
     \\
   \hline
\rowcolor{lightgray}

      \# & Description & Source & Priority & Version \\ \hline
      E1 & \textbf{Ability to determine response of Building Subject to Earthquake hazard including formal treatment of randomness and uncertainty uncertainty} & GC & M  & 1.0  \\ \hline
 E1.1 & Ability of Practicing Engineers to use multiple coupled resources (applications, databases, viz tools) in engineering practice & GC & M 1.0 & \\ \hline
E1.2 & Ability to utilize resources beyond the desktop including HPC & GC & M & 1.0 \\ \hline
E1.3 & Tool should incorporate data from www & GC & M & 1.0 \\ \hline
E1.4 & Tool available for download from web & GC & M & 1.0 \\ \hline
E1.5 & Ability to benefit from programs that move research results into practice and obtain training & GC & M & \\ \hhline{=====}

      E2 & \textbf{Ability to select from different Input Motion Options} & SP & M & 1.0  \\ \hline
      E2.1 & Ability to select from Multiple input motions and view UQ due to all the discrete events & GC & M & 1.0  \\ \hline
      E2.2 & Ability to select from list of SimCenter motions & SP & M & 1.0 \\ \hline
      E2.3 & Ability to select from list of PEER motions & SP & D & 1.0 \\ \hline
      E2.4 & Ability to use OpenSHA and selection methods to generate motions & UF & D & 1.0 \\ \hline
      E2.5 & Ability to Utilize Own Application in Workflow & SP & M & 1.0 \\ \hline
      E2.6 & Ability to use Broadband & SP & D &  \\ \hline
      E2.7  & Ability to include Soil Structure Interaction Effects & GC & M & 1.1 \\  \hline
      E2.7.1  & 1D nonlinear site response with effective stress analysis & SP & M & 1.1  \\ \hline
      E2.7.2  & Nonlinear site response with bidirectional loading & SP & M & 1.2 \\  \hline
      E2.7.3  & Nonlinear site response with full stochastic characterization of soil layers & SP & M &  \\ \hline
      E2.7.4 & Nonlinear site response, bidirectional different input motions  & SP & M &  \\  \hline
      E2.7.5 & Ability to couple models from point of rupture through rock and soil into structure, which represents future of professional design practice & GC & M &  \\  \hline
      E2.7.5.1 & Interface using DRM method  & SP  & M &  \\  \hline
      E2.8 & Utilize PEER NGA www ground motion selection tool  & UF & D & 2.0 \\ \hline
      E2.9 & Ability to select from synthetic ground motions & SP & M & 1.0  \\
      E2.9.1 & per Vlachos, Papakonstantinou, Deodatis (2017) & SP & D & 1.1  \\ 
      E2.9.2 & per Dabaghi, Der Kiureghian (2017) & UF & D & 2.0 \\  \hhline{=====}
      
      
	E3 & \textbf{Ability to select different Building Model Generators} & GC & M & 1.0 \\ \hline
	E3.1 & Ability to quickly create a simple nonlinear building model for simple methods of seismic evaluation & GC 2.T13 & D & 1.1 \\ \hline
	E3.2 & Ability to use existing OpenSees model scripts & SP & M & 1.0 \\ \hline
	E3.3  & Ability to define building and use Expert System to generate FE mesh & SP & &  \\ \hline
	E3.3.1 & Expert system for Concrete Shear Walls & SP & M &  \\ \hline
	E3.3.2 & Expert system for Moment Frames & SP & M &  \\ \hline
	E3.3.3 & Expert system for  Braced Frames & SP & M &   \\ \hline
	E3.4 & Ability to define building and use Machine Learning applications to generate FE & GC &  &  \\ \hline
	E3.4.1 & Machine Learning for Concrete Shear Walls & SP & M &  \\ \hline
	E3.4.2 & Machine Learning for Moment Frames & SP & M &  \\ \hline
	E3.4.3 & Machine Learning for Braced Frames & SP & M &   \\ \hline
	E3.5 & Ability to specify connection details for member ends & UF & M & 2.2 \\ \hline
	E3.6 & Ability to define a user-defined moment-rotation response representing the connection details & UF & D & 2.2 \\  \hhline{=====}
	
	
	E4 & \textbf{Ability to select from different Nonlinear Analysis options} & GC & M & 1.0 \\ \hline
	E4.1 & Ability to specify OpenSees as FEM engine and to specify different analysis options & SP & M & 1.0 \\ \hline
	E4.2 & Ability to provide own OpenSees Analysis script to OpenSees engine. & SP & D & 1.0 \\ \hline
	E4.3 & Ability to provide own Python script and use OpenSeesPy engine. & SP & O & 1.2 \\ \hline
	E4.4 & Ability to use alternative FEM engine. & SP & M & 2.0 \\ \hhline{=====}

    E5 & \textbf{Engineering Demand Parameters} & SP & M &  \\ \hline
    E5.1 & Ability to specify standardized set of outputs & SP & M & 1.0  \\ \hline
    E5.2 & Ability to Process own Output Parameters & UF & M & 1.1  \\ \hline
    E5.3 & Add to Standard Earthquake a variable indicating analysis failure & UF & D &   \\ \hhline{=====}


UQ & \textbf{Ability to use various UQ Methods} & GC & M &  \\ \hline
UQ1 & \textbf{Forward Propagation Methods} & GC  & M & 1.0 \\ \hline
UQ1.1 & Ability to use basic Monte Carlo and LHS methods & SP & M & 1.0 \\ \hline
UQ1.2 & Ability to use Importance Sampling  & SP & M & 2.0 \\ \hline
UQ1.3 & Ability to use Gaussian Process Regression & SP & M & 2.0 \\ \hline
UQ1.4 & Ability to use Own External UQ Engine & SP & M &  \\ \hline
UQ2 & \textbf{Ability to use various Reliability Methods} & UF & M & 1.0 \\ \hline
UQ2.1 & Ability to use First Order Reliability method & UF & M &  \\ \hline
UQ2.2 & Ability to use Second Order Reliability method & UF & M & \\ \hline
UQ2.2 & Ability to use Surrogate Based Reliability & UF & M & \\ \hline
UQ2.3 & Ability to use Own External Application to generate Results & UF & M &  \\ \hline
UQ3 & \textbf{Ability to user various Sensitivity Methods} & UF & M & 1.0  \\ \hline
UQ3.1 & Ability to obtain Global Sensitivity Sobol's indices & UF & M &  \\ \hline
UQ4 & \textbf{Various Random Variable Probability Distributions} & SP & M & 1.0 \\ \hline
UQ4.1 & Ability to Define Variables of different types: & SP & M & 1.0  \\ \hline
UQ4.1.1 & Normal & SP & M  & 1.0 \\ \hline
UQ4.1.2 & Lognormal & SP & M & 1.0 \\ \hline
UQ4.1.3 & Uniform & SP & M & 1.0  \\ \hline
UQ4.1.4 & Beta & SP & M & 1.0 \\ \hline
UQ4.1.5 & Weibull &  SP & M  & 1.0 \\ \hline
UQ4.1.6 & Gumbel &  SP & M & 1.0  \\ \hline
UQ4.2 & User defined Distribution & SP & M &  \\ \hline
UQ4.3 & Define Correlation Matrix & SP & M &  \\ \hline
UQ4.4 & Random Fields & SP & M &  \\ \hline
>>>>>>> b55c16b9402fc3444addf52182320452de7e27c2



    EE & \textbf{Ability to obtain Educational material} &  &  & \\ \hline
    EE1 & Ability to use educational provisions to gain interdisclipinary education so as to gain expertise in earth sciences and physics, engineering mechanics, geotechnical engineering, and structural engineering in order to be qualified to perform these simulations & GC & D & \\ \hline
    EE2 & Documentation exists on tool usage & SP & M & 1.1  \\ \hline
    EE3 & Video Exists demonstrating usage & SP & M & 1.1  \\ \hline
    EE4 & Verification Examples Exist & SP & M & 1.1  \\ \hhline{=====}
    WE5 & Validation Examples Exist, validated against tests or other software & GC & M &  \\\hhline{=====}
    EM & \textbf{Misc.} &  & & \\ \hline
    EM1 & Tool to allow user to load and save user inputs & SP & M & 1.0 \\ \hline
    EM2 & Add to Standard Earthquake a variable indicating analysis failure & UF & D &   \\ \hline
    EM3 & Installer which installs application and all needed software & UF & M &   \\ \hline
      
  \bottomrule 
               
\end{longtable}

\noindent
KEY:\\
Source: GC=Needed for Grand Challenges, SP=Senior Personnel, UF=User Feedback \\
Need: M=Mandatory, D=Desirable, P=Possible Future \\
Version: Version number the basic requirement was met 


\clearpage
\subsection{quoFEM}
The following are the requirements are being for the quoFEM application. quoFEM is an application proving UQ and Optimization methods to existing FEM applications. uqFEM has a lower level interface to UQ and Optimization methods than the other applications (WE-UQ, EE-UQ, and PBE). It is thus a more powerful tool providing more capabilities for researchers. All requirements in this section are related to work in WBS 1.3.8.

 \begin{longtable}{| p{.07\textwidth} | p{.65\textwidth} | p{.08\textwidth} | p{.08\textwidth} |  p{.08\textwidth} |}
\caption{Reuirements for quoFEM}             
  \label{tab:featureQuo_FEM}     
     \\
   \hline
\rowcolor{lightgray}
      \# & Description & Source & Priority & Version \\ \hline
      Q1 & A tool that facilitates civil engineers performing UQ methods and provides a platform for those in UQ to promote their research to a broader audience & SP & M & 1.0     \\ \hline
      Q1.1 &  Ability to utilize existing open-source software for faster deployment
 & SP & M & 1.0 \\ \hline    
     Q1.2 & Ability to extend to multiple UQ engines" & SP & M & \\ \hline
     Q1.3 & Ability to utilize resources beyond the desktop, e.g. HPC, for computations & SP & M & 1.0 \\  \hhline{=====}
     Q2 & \textbf{Forward Uncertainty Propagation} & SP &  &  \\  \hline
	Q121 & Input uncertainty characterization & SP & M & 1.1 \\ \hline
	Q2.2 & PDF Approximation & SP & M & 1.1 \\ \hline
	Q2.3 &  Descriptive output statistics & SP & M & 1.1 \\ \hline
	Q2.4 &  Basic Monte Carlo Sampling  & SP & M & 1.1 \\ \hline	
	Q2.5 &  Importance Sampling for rare events  & SP & M & 2.0 \\ \hline	
	Q2.6 &  Cross-Entropy sampling  & SP & M &  \\ \hline
	Q2.7 &  Forward Propagation, GPR Surrogate  & SP & M & 2.0 \\ \hline
	Q2.8 &  Forward Propagation, PCE Surrogate  & SP & M & 2.0 \\ \hline
	Q2.9 &  Multi-fidelity sampling  & SP & M &  \\ \hline
	Q2.10 &  Spatial/temporal stochastic models  & SP & M &  \\ \hhline{=====}
	Q3 & \textbf{Sensitivity Analysis} &  &  & \\ \hline
	Q3.1 & Global sensitivity Sobol's indices & SP & M & 2.0  \\ \hhline{=====}
	Q4 & \textbf{System Identification and Bayesian Inference} &  &  \\ \hline
	Q4.1 & Parameter estimation & SP & M &  \\ \hline
	Q4.2 & Basic Bayesian parameter updating & SP & M &  \\ \hline
	Q4.3 & Advanced MCMC-based Bayesian updating & SP & M & \\ \hline
	Q4.4 & Advanced Surrogate-based Bayesian updating & SP & M &  \\ \hline
	Q4.5 & Model class selection & SP & M &  \\ \hline
	Q4.6 & Sequential Bayesian updating & SP & M &  \\ \hhline{=====}
	Q5 & \textbf{Optimization under Uncertainty} &  &  & \\ \hline
	Q5.1 & Reliability-Based Design Optimization & SP & M &  \\ \hline
	Q5.2 & Single-objective optimization under uncertainty & SP & M &  \\ \hline
	Q5.3 & Multi-objective optimization under uncertainty & SP & M &  \\ \hhline{=====}
	Q6 & \textbf{Reliability Analysis} &  &  & \\ \hline
	Q6.1 & First/Second Order Reliability Methods & SP & M & 2.0 \\ \hline
	Q6.2 & Surrogate-based reliability & SP & M & 2.0 \\ \hline
    QE & \textbf{Ability to obtain Educational material} &  SP &  M & 1.0  \\ \hline
    QE1 & Documentation exists on tool usage & SP & M & 1.0 \\ \hline
    QE2 & Video Exists demonstrating usage & SP & M & 1.0 \\ \hline
    QE3 & Verification Examples Exist & SP & M &   \\ \hline
	\bottomrule
\end{longtable}

\noindent
KEY:\\
Source: GC=Needed for Grand Challenges, SP=Senior Personnel, UF=User Feedback \\
Need: M=Mandatory, D=Desirable, P=Possible Future \\
Version: Version number the basic requirement was met 




\clearpage
\subsection{Performance Based Engineering}
The following are the requirements for application(s) related to performance based engineering of a single structure related to natural hazards such as earthquake and hurricane . The requirements are being met by the PBE application. All requirements in this section are related to work in WBS 1.3.9.

\begin{longtable}{| p{.07\textwidth} | p{.65\textwidth} | p{.08\textwidth} | p{.08\textwidth} |  p{.08\textwidth} |}

\caption{Requirements for PBE}
  \label{tab:featureRequirements}   
    \\
   \hline
\rowcolor{lightgray}
      \# & Description & Source & Priority & Version \\ \hline
      
      P1 & \textbf{Ability to determine damage and loss calculations for a building subjected to a natural hazard including formal treatment of randomness and uncertainty uncertainty} & GC & M  & 1.0  \\ \hline
      P1.1 & Ability to determine  damage and loss for multiple different hazards & GC & M &  \\ \hline
      P1.1.1 & Damage and Loss for ground shaking due to Earthquake & GC & M & 1.0 \\ \hline
      P1.1.2 & Damage and Loss due to Wind Loading & GC & M &  \\ \hline
      P1.1.3 & Damage and Loss due to water damage due to Tsunami or Coastal Inundation & GC & M &  \\ \hline
     P1.1 & Ability of Practicing Engineers to use multiple coupled resources (applications, databases, viz tools) in engineering practice & GC & 1.0 \\ \hline
P1.2 & Ability to utilize resources beyond the desktop including HPC & GC & M & 1.0 \\ \hline
P1.3 & Tool should incorporate data from WWW & GC & M & 1.0 \\ \hline
P1.4 & Tool available for download from web & GC & M & 1.0 \\ \hline
P1.5 & Ability to use new viz tools for viewing large datasets generated by PBE & GC & M & 1.0 \\ \hhline{=====}
P2 & \textbf{Various Motion Selection Options for Hazard Event} & SP & M & 1.0  \\ \hline

P2.1 & Various Earthquake Events & SP & M & 1.0  \\ \hline
\softwareSwitch{PBE}{ 
 P2.1.1 & Ability to select from Multiple input motions and view UQ due to all the discrete events & GC & M & 1.0  \\ \hline
 P2.1.2 & Ability to select from list of SimCenter motions & SP & M & 1.0 \\ \hline
 P2.1.3 & Ability to select from list of PEER motions & SP & D & 1.0 \\ \hline
 P2.1.4 & Ability to use OpenSHA and selection methods to generate motions & UF & D & 1.0 \\ \hline
 P2.1.5 & Ability to Utilize Own Application in Workflow & SP & M & 1.0 \\ \hline
 P2.1.6 & Ability to use Broadband & SP & D &  \\ \hline
 P2.1.7  & Ability to include Soil Structure Interaction Effects & GC & M & 1.1 \\  \hline
 P2.1.7.1  & 1D nonlinear site response with effective stress analysis & SP & M & 1.1  \\ \hline
 P2.1.7.2  & Nonlinear site response with bidirectional loading & SP & M & 1.2 \\  \hline
 P2.1.7.3  & Nonlinear site response with full stochastic characterization of soil layers & SP & M &  \\ \hline
 P2.1.7.4 & Nonlinear site response, bidirectional different input motions  & SP & M &  \\  \hline
 P2.1.7.5 & Building in nonlinear soil domain utilizing large scale rupture simulation & GC  & M &  \\  \hline
      P2.1.7.5.1 & Interface using DRM method  & SP  & M &  \\  \hline
      P2.1.8 & Utilize PEER NGA www ground motion selection tool  & UF & D & 2.0 \\ \hline
      P2.1.9 & Ability to select from synthetic ground motions & SP & M & 1.0  \\
      P2.1.9.1 & per Vlachos, Papakonstantinou, Deodatis (2017) & SP & D & 1.1  \\ 
      P2.1.9.2 & per Dabaghi, Der Kiureghian (2017) & UF & D & 2.0 \\ \hline
}{
 P2.1  & Ability to select from all EE-UQ Event Options listed in EE-UQ E2& SP & M & 1.0  \\ \hline 
}   


 P2.2 & Various Wind Loading Options & SP & M &   \\ \hline
 \softwareSwitch{PBE}{ 
 } {
 P2.1  & Ability to select from all WE-UQ Event Options listed in WE-UQ W2 & SP & M & 1.0  \\ \hline   
 }
 
 \softwareSwitch{PBE}{ 
 P2.3 & Various Water Loading Options & SP & M &   \\ \hhline{=====}
 } {
 P2.3 & Various Water Loading Options & SP & M &   \\ \hline
 P2.3.1 & Ability to select from all HydroUQ Event Options & SP & M & 1.0  \\ \hhline{=====}
 }
 
 
 P3 & \textbf{Building Model Generation} & GC & M &  \\ \hline
  \softwareSwitch{PBE}{ 
  P3.1 & Ability to quickly create a simple nonlinear building model & GC & D & 1.1 \\ \hline
  P3.2 & Ability to use existing OpenSees model scripts & SP & M & 1.0 \\ \hline
  P3.3  & Ability to define building and use Expert System to generate FE mesh & SP & &  \\ \hline
  P3.3.1 & Expert system for Concrete Shear Walls & SP & M &  \\ \hline
  P3.3.2 & Expert system for Moment Frames & SP & M &  \\ \hline
  P3.3.3 & Expert system for  Braced Frames & SP & M &   \\ \hline
  P3.4 & Ability to define building and use Machine Learning applications to generate FE & GC &  &  \\ \hline
  P3.4.1 & Machine Learning for Concrete Shear Walls & SP & M &  \\ \hline
  P3.4.2 & Machine Learning for Moment Frames & SP & M &  \\ \hline
  P3.4.3 & Machine Learning for Braced Frames & SP & M &   \\ \hline
  P3.5 & Ability to specify connection details for member ends & UF & M &  \\ \hline
  P3.6 & Ability to define a user-defined moment-rotation response representing the connection details & UF & D & 2.2 \\ \hhline{=====}
  } {  
 P3.1 & Ability to Select All Building Model Generators in EE-UQ & SP & M & 1.0 \\ \hhline
 P3.2 & Ability to Select All Building Model Generators in WE-UQ & SP & M &  \\ \hhline
 P3.3 & Ability to Select All Building Model Generators in HydroUQ & SP & M &  \\ \hhline{=====}
  }
 

  P4 & \textbf{Perform Nonlinear Analysis} & GC & M & 1.0 \\ \hline
  P4.1 & Ability to specify OpenSees as FEM engine and to specify different analysis options & SP & M & 1.0 \\ \hline
  P4.2 & Ability to provide own OpenSees Analysis script to OpenSees engine. & SP & D & 1.0 \\ \hline
  P4.3 & Ability to provide own Python script and use OpenSeesPy engine. & UF & O &  \\ \hline
  P4.4 & Ability to use alternative FEM engine. & SP & M & 2.0 \\ \hhline{=====}
  
   P5 & \textbf{Uncertainty Quantification Methods} &  GC & M & 1.0  \\ \hline
   \softwareSwitch{PBE}{ 
  P5.1 & \textbf{Various Forward Propogation Methods} & SP & M & 1.0  \\ \hline
  P5.1.1 & Ability to use basic  Monte Carlo and LHS methods & SP & M & 1.0 \\ \hline
  P5.1.2 & Ability to use Importance Sampling  & SP & M & 2.0 \\ \hline
  P5.1.3 & Ability to use Gaussian Process Regression & SP & M & 2.0 \\ \hline
  P5.1.4 & Ability to use Own External UQ Engine & SP & M &  \\ \hline
  P5.2 & \textbf{Various Reliability Methods} & UF & M &  \\ \hline
  P5.2.1 & Ability to use First Order Reliability method & UF & M &  \\ \hline
  P5.2.2 & Ability to use Second Order Reliability method & UF & M & \\ \hline
  P5.2.2 & Ability to use Surrogate Based Reliability & UF & M & \\ \hline
  P5.2.3 & Ability to use Own External Application to generate Results & UF & M &  \\ \hline
  P5.3 & \textbf{Various Sensitivity Methods} & UF & M &  \\ \hline
  P5.3.1 & Ability to obtain Global Sensitivity Sobol's indices & UF & M &  \\ \hline
  } {
  P5.1 & Ability to use all forward propogation methods available in EE-UQ and WE-UQ  section U1 & SP & M & 1.0 \\ \hhline{=====}
  }
  
   P6 & \textbf{Random Variables for Uncertainty Quantification} & GC & M & 1.0  \\ \hline
   
    \softwareSwitch{PBE}{ 
   P6.1 & Ability to Define Variables of certain types: & SP & M & 1.0  \\ 
   P6.1.1 & Normal & SP & M  & 1.0 \\ \hline
    P6.1.2 & Lognormal & SP & M & 1.0 \\ \hline
    P6.1.3 & Uniform & SP & M & 1.0  \\ \hline
    P6.1.4 & Beta & SP & M & 1.0 \\ \hline
    P6.1.5 & Weibull &  SP & M  & 1.0 \\ \hline
    P6.1.6 & Gumbel &  SP & M & 1.0  \\ \hline
    P6.2 & User defined Distribution & SP & M &  \\ \hline
    P6.3 & Correlated Random Variables & SP & M &  \\ \hline
    P6.4 & Random Fields & SP & M &  \\ \hhline{=====}
    } {
    P6.1 & Ability to use all random variable distributions in EE-UQ and WE-UQ  section U4 & SP & M & 1.0 \\ \hhline{=====}
    }
    
    
   P8 & \textbf{Engineering Demand Parameters} &  &  & \\ \hline
    P8.1 & Ability to Process own Output Parameters & UF & M &   \\ \hline
    P8.2 & Add to Standard Earthquake a variable indicating analysis failure & UF & D &   \\ \hline
    P8.3 & Allow users to provide their own set of EDPs for the analysis. & UF & D & 2.0\\ \hline

P9 & \textbf{Damage and Loss Assessment} & GC & M & 1.0 \\ \hline
P9.1 & Different Assessment Methods & GC & M & 2.0 \\ \hline
P9.1.1 & Ability to perform component-based (FEMA P58) loss assessment for an earthquake hazard. & SP & M & 1.0 \\ \hline
P9.1.2 & Ability to perform component-assembly-based (HAZUS MH) loss assessment for an earthquake hazard. & SP & D & 1.1 \\ \hline
P9.1.3 & Ability to perform downtime estimation using the REDi methodology. & UF & D & \\ \hline
P9.1.4 & Ability to describe building performance with additional decision variables from HAZUS (e.g., business interruption, debris) & SP & D &  \\ \hline
P9.1.5 &  Ability to perform time-based assessment & GC & M &  \\ \hline
P9.1.6 & Ability to perform damage and loss assessment for hurricane wind & GC & M &  \\ \hline
P9.1.7 & Ability to perform damage and loss assessment for storm surge & GC & M &  \\ \hline
P9.2 & Control & SP & M & 1.0 \\ \hline
P9.2.1 & Allow users to set the number of realizations & SP & M & 1.0\\ \hline
P9.2.2 &  Allow users to specify the added uncertainty to EDPs & SP & M & 1.0 \\ \hline
P9.2.3 &  Allow users to decide which decision variables to calculate & SP & D & 1.0 \\ \hline
P9.2.4 &  Allow users to set the number of inhabitants on each floor and customize their temporal distribution. & SP & D & 1.0 \\ \hline
P9.2.5 &  Allow users to specify the boundary conditions of repairability. & SP & D & 1.0 \\ \hline
P9.2.6 &  Allow users to control collapse through EDP limits. & SP & D & 1.0\\ \hline
P9.2.7 &  Allow users to specify the replacement cost and time for the building. & SP & M & 1.0 \\ \hline
P9.2.8 &  Allow users to specify EDP boundaries that correspond to reliable simulation results. & SP & D & 1.0\\ \hline
P9.2.9 & Allow users to specify collapse modes and characterize the corresponding likelihood of injuries. & SP & D & 1.0\\ \hline
P9.2.10 & Allow users to specify the collapse probability of the structure. & UF & M & 1.2\\ \hline
P9.2.11 & Allow users to use empirical EDP data to estimate the collapse probability of the structure. & UF & M & 1.2\\ \hline
P9.2.12 & Allow users to choose the type of distribution they want to estimate the EDPs with. & UF & D & 1.2\\ \hline
P9.2.13 & Allow users to perform the EDP fitting only for non-collapsed cases. & UF & M & 1.2\\ \hline
P9.2.14 & Allow users to couple response estimation with loss assessment. & UF & M & \\ \hline
P9.3 & Component damage and loss information & SP & M & 1.0\\ \hline
P9.3.1 & Make the component damage and loss data from FEMA P58 available. & SP & M & 1.0 \\ \hline
P9.3.2 & Ability to use custom components for loss assessment. & SP & D & 1.0 \\ \hline
P9.3.3 & Allow users to set different component quantities for each floor in each direction. & SP & D & 1.0 \\ \hline
P9.3.4 & Allow users to set the number of identical component groups and their quantities within each performance group. & UF & D & 1.0 \\ \hline
P9.3.5 & Use a generic JSON data format for building components that can be shared by component-based and component-assembly-based assessments. & SP & D & 1.1 \\ \hline
P9.3.6 & Convert FEMA P58 and HAZUS component damage and loss data to the new JSON format and make it available with the tool. & SP & D & 1.1 \\ \hline
P9.3.7 & Make component definition easier by providing a list of available components in the given framework (e.g. FEMA P58 or HAZUS) and not requesting inputs that are already available in the data files. & UF & D & 1.2 \\ \hline
P9.3.8 & Make the component damage and loss data from FEMA P58 2nd edition available. & UF & M & 2.0 \\ \hline
P9.3.9 & Improve component definition by providing complete control over every characteristic on every floor and in every direction & UF & D & 2.0 \\ \hline
P9.3.10 & Allow users to view fragility and consequence functions in the application & UF & D &  \\ \hline
P9.3.11 & Allow users to edit fragility and consequence functions in the application & UF & D &  \\ \hline
P9.4 & Stochastic loss model & SP & M & 1.0 \\ \hline
P9.4.1 & Allow the user to specify basic dependencies (i.e. independence or perfect correlation) between logically similar parts of the stochastic model (i.e. within component quantities or one type of decision variable, but not between quantities and fragilities) & SP & D & 1.0 \\ \hline
P9.4.2 & Allow the user to specify basic dependencies between reconstruction cost and reconstruction time. & SP & D & 1.0 \\ \hline
P9.4.3 & Allow the user to specify basic dependencies between different levels of injuries. & SP & D & 1.0 \\ \hline
P9.4.4 & Allow the user to specify intermediate levels of correlation (i.e. not limited to 0 or 1) and provide a convenient interface that makes sure the specified correlation structure is valid. & SP & D & \\ \hline   
P9.4.5 & Allow the user to specify the correlation for EDPs. & SP & D &  \\ \hhline{=====}

 PM & \textbf{Misc.} & UF & M & 1.2  \\ \hline
    PM.1 & Tool to allow user to load and save user inputs & SP & M & 1.0 \\ \hline
   PM.2 & Simplify run local and run remote by removing workdir locations. Move to preferences & UF & D & 1.2  \\ \hline
  P M.3 & Add to EDP a variable indicating analysis failure & UF & D &   \\ \hline
   PM.4 & Installer which installs application and all needed software & UF & M &   \\ \hhline{=====}
 PE & \textbf{Ability to gain educational materials that will help and encourage PBE} & GC  & M & 1.0 \\ \hline
 PE.1 & Documentation exists on tool usage & SP & M & 1.1  \\ \hline
 PE.2 & Video Exists demonstrating usage & SP & M & 1.1  \\ \hline
 PE.3 & Verification Examples Exist & SP & M & 1.1  \\ \hline
 PE.4 & Validation Examples Exist, validated against tests or other software & GC & D &  \\ \hline
  \bottomrule               
\end{longtable}


\noindent
KEY:\\
Source: GC=Needed for Grand Challenges, SP=Senior Personnel, UF=User Feedback \\
Need: M=Mandatory, D=Desirable, P=Possible Future \\
Version: Version number the basic requirement was met 

 


\nocite{*}

% \appendix
% \chapter{More Monticello Candidates}

\pagestyle{plain}
{
  \renewcommand{\thispagestyle}[1]{}	
  \printbibliography           
}

\end{document}
